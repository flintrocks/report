We are happy with what we have achieved in terms of features implemented on the existing compiler. Overall, we believe the project was a success and we hit all of the goals agreed upon with our supervisor and added further extensions to improve the quality of code and development workflow.

We think that our rigorous process helped to keep everything in order and allowed us to move towards our goals quickly. One of the most important parts of the process were the daily stand-ups on Slack: we think it played a major role in keeping every member of the team in sync. The Kanban method allowed us to keep issues flowing through the stages of development with little friction. Our Kanban board paired well with the GitHub Slack integration, ensuring that issues did not wait for a code review for too long. Some of the members of the team had no previous practical experience with Agile development beyond the Software Engineering classes and this project presented itself as an opportunity to get experience an Agile development process.

We feel that throughout this project we learnt a lot of things, both technical and non-technical. One of the most important topics we gathered a lot of insight about is the Ethereum ecosystem, which few of the team members had in-depth knowledge of when we started. The nature of the project required that we had to get familiar with all aspects of the smart contract development process including the lower-level details of the ecosystem, such as Solidity, YUL, and the EVM. Only two members of the team had previous experience with Swift, so the others had the chance to learn about Swift Programming.

At the start of the project, we decided that we will run code reviews for every merge request and rigorously kept doing so throughout the project, with some reviews becoming extremely long discussions with multiple stages of change requests. This not only helped us to ensure a high code quality standard, but we believe it also played a role in improving our own code style and learning good practices from our teammates.

We enjoyed the iterations and regularly meeting with our supervisor, which felt like working for a client, with everything that involves including meeting deadlines and the pressure of developing a good product.

\section{Extensions}

There are many parts of the compiler that can be improved and features that are desirable to have. These are just a few:

\begin{itemize}
	\item Modularisation\\
Adding modules allows a contract definition to be split across multiple source files and for the inclusion and distribution of user-made libraries and frameworks.
	\item Linear types\\
Linear types are a more integrated version of the Asset traits that we have implemented. These types might be implemented with additional syntax and memory operations, allowing variables that hold quantities of an asset that can be split atomically, never duplicated, and never destroyed. For example, with an asset such as Wei in the context of a @payable contract function, this means we add a semantic check that ensures that the entire value of the incoming asset value is transferred to another instance to prevent its destruction.
	\item Gas asset\\
A built-in Asset type, similar to Wei, that is used for the `gas` hyper-parameter of an external call.
	\item Generics\\
At some point it may be desirable for the Flint language to feature generics. This feature requires careful design and consideration to stay compatible with Solidity contracts. Options include allowing generics only internally to a Flint contract, developing a Flint ABI that allows generics and is independent of Solidity, or encoding generics into the Solidity ABI itself.
	\item Flint-contract interop without Solidity ABI\\
The Solidity ABI severely limits Flint's capabilities to provide safer functionality when interoperating with other contracts. A Flint-specific ABI would allow us to encode additional information about types (such as type states, generics, exceptions) and introduce a trust model for external contracts.
	\item Targeting eWASM\footnote{\url{https://github.com/ewasm/design}}\\
eWASM is a new Ethereum backend, set to replace EVM in the future. By targeting eWASM and potentially replacing the YUL backend keeps Flint up to date with current developments in the Ethereum community and makes it future-proof. With WASM being an industry standard backed by large players in the tech industry, it is poised to become widely implemented, allowing Flint to run on more platforms.
	\item Flint package manager\footnote{\url{https://github.com/flintlang/flint/pull/151/files}}\\
A package manager has long been a requested feature, but until we implemented external calls there was no reason to have one because there would've been no way to call another contract. Additionally, since the language does not currently support modules, there is no way to import code from another Flint program. Once modules and a Flint ABI are supported, there will be a better case for having a package manager.
	\item Contract trust model\\
Alongside the package manager, we also wanted to support a trust model between contracts. Under this model, a distinction is made between Flint contracts that have been verified and are guaranteed to work safely and contracts that have issues causing them to misbehave. The compiler might then refuse to compile code that is calling an unsafe contract without taking necessary precautions such as handling an error case.
	\item LiquidHaskell-style verification support\footnote{\url{https://ucsd-progsys.github.io/liquidhaskell-blog/}}\\
LiquidHaskell defines logical predicates that allow the programmer to enforce properties at compile-time. In line with the safe programming aspirations of the Flint language, an extension to the compiler is to support predicate annotations that are similar in function and syntax to aid the compiler in verifying correctness and support the issuance of certificates of safeness.
\end{itemize}

We believe that with our contributions, future development will be easier and also less error-prone. With unit testing set up, we have created a solid foundation for future development. Due to time constraints, we were not able to write unit tests for every part of the compiler. Instead, we chose to adopt an incremental technique, writing unit tests along the way as we developed or changed parts of the code. In future development and as an extension to the compiler, the test suite should cover the code base fully.

Another area of improvement is the diagnostics emitted during compilation, which could benefit from more explicit and descriptive semantic analysis errors. The way that compiler passes are currently optimised is not optimal and makes unit testing difficult. A possible improvement is to split the passes based on their function even further and into separate files and units to make them more modular and easier to test. To aid with programmer reasoning, the semantic analyser pass could also be reduced to the application of a series of rules, that match properties of the AST using a domain specific language similar to XPath\footnote{\url{https://en.wikipedia.org/wiki/XPath}}. This allows semantic checks to be categorised in code and kept separate from other, unrelated semantic checks.

A further extension to the editor support for Flint would be to allow a wider range of editors to be used. Because we have integrated the LSP with the Flint compiler, we can support any editor that supports the LSP, including Eclipse, IntelliJ, and Atom\footnote{\url{https://langserver.org}}. Extending editor support is a matter of creating a small shim plugin for each editor that automatically manages the lifetime of the language server.

Automatic code completion is another desirable feature for the Flint language editor support. Adding automatic code completion to the compiler's language server is a large undertaking as we would need to create a parser that supports creating partial ASTs, as well as create and maintain more semantic information than we currently do. In order to pass this semantic information to the editor it would be necessary to refactor large parts of the existing codebase.

\section{Ethics}

Ethical issues are an important part to consider in every project and we carefully considered them throughout development.

One of the most critical issues is claiming Flint is `safe by design', which requires a lot of responsibility. The main purpose of the Flint project is to build a language that is strict enough to act as a safety net for the programmer. Since contracts written in Flint handle real assets with monetary value, we cannot afford any compilation or runtime flaws that contradict the spirit and existing documentation of the language. During our development, we discovered instances of such bugs in the compiler when trying to implement new features. In order to support the safety claim, we decided we needed stringent unit testing.

The compiler is run on the user's machine, so once distributed, the user is responsible for its usage. We do not collect any data at all, so ethical issues regarding data collection and privacy cannot apply. Another thing to note is the fact that the compiler is open source, so it is not just a ‘black box’. Everyone can audit the code and compile their own version of the project, having total control.

Other ethical issues that we considered are those that do not directly relate to Flint as a programming language, but the entire Ethereum ecosystem. Important ethical issues concerning the ecosystem more generally include the environmental impact of the blockchain and the anonymity of deployed contracts.

Blockchain networks have a significant environmental impact and with Ethereum\cite{mb-electricity} being one of them, it is no exception. Greener alternatives are being developed, although few are as successful as Bitcoin or Ethereum. Unlike Bitcoin, the Ethereum foundation has a plan to reduce the environmental impact of its mining algorithm\cite{cd-pos}. Importantly, Flint is not necessarily tied to the Ethereum ecosystem and could be retargeted in the future to work on other, greener alternative platforms.

Anonymity of deployed contracts is also a significant issue, since everything is public on the blockchain. That does not play well with certain domains like the health and financial sectors, which have very high privacy requirements. Solutions\cite{bnc-mobius} to add a layer of privacy to cryptocurrencies currently exist and significant work is being put into this area. One of these solutions was brought in by ZCash, Ethereum bringing support for one of their features with their October 2017 update\cite{bnc-mobius}. ZCash implements something called 'Zero-Knowledge Succinct Non-Interactive Argument of Knowledge', or zk-Snark for short, based on the concept of zero-knowledge cryptography. Zero-knowledge cryptography means that the proof of possession of some information can be verified without any interaction between the prover and the verifier, therefore not revealing anything\cite{zkskarks}.
