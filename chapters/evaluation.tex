In this project we added external calls to Flint, making the language more useful for real-world Ethereum contract use. Franklin's Flint could not compile many real-world contracts because interoperating with the outside world was necessary. An example contract that might crop up in the real world can be found in the appendix.

Another one of our major contributions to the compiler was improving the test suite. In particular, we added a unit testing framework, as detailed in chapter \ref{chp:di}. Although using this framework to its full potential by refactoring each module to be unit testable (and hence of better code quality) was out of scope of our project, this change will allow improvements of the code base by future contributors.

The other, pre-existing part of the test suite were the integration tests. Throughout our project we made sure that any changes passed all the behaviour tests on the CI server prior to merging them into the protected master branch. The integration tests verified the compiler’s three key phases:

\begin{itemize}
	\item Parser tests – verify that code was parsed into the correct AST.
	\item Semantic tests – verify that the proper semantic errors were emitted where expected.
	\item Behaviour tests – verify that the contracts produced by the compiler work correctly in an actual EVM environment (tested with the Truffle\footnote{\url{https://truffleframework.com/}} suite).
\end{itemize}

Each language feature we added into Flint was also accompanied by relevant parser, semantic, and behaviour tests. These tests were usually written prior to the specific feature implementation, e.g. in the ticket description. The ticket itself was created by whoever took the original analysis ticket and any further elaboration was then expected from them. The ticket description with accompanying code fragments and tests served as a kind of specification for the new feature, emulating TDD practices.

In terms of practical usability, Flint was greatly improved with the introduction of external calls, also detailed in chapter \ref{chp:di}. There still exist some limitations due to the intentionally simple type system in Flint, so a mapping from any Solidity type to a Flint type is not always possible. However, we have demonstrated calls to Solidity contracts in the Truffle testing environment, as well as the Ethereum Remix IDE\footnote{\url{https://remix.ethereum.org/}}.

Our integration with the Language Server Protocol opens up Flint to new developers and make the language more approachable than ever. Receiving instantaneous feedback on code correctness is crucial to the user’s understanding of the language and our integration with Visual Studio Code in particular makes writing smart contracts in Flint easier than before.
