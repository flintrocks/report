Ethereum is an open source platform that enables developers to build and deploy decentralised applications by leveraging blockchain technology\cite{bg-ethereum}. In short, the blockchain enables a decentralised trust model due to computation being intrinsically tied to the progression of time by using cryptographic hashes\cite{ptb}. Past blocks cannot be altered and all nodes will eventually be consistent with each other. A currency, Ether, drives the network to incentivise users to keep it running. Ether is created by nodes that successfully verify blocks on the network (called mining) and can be transferred as part of a transaction on the ledger. Wei is a smaller denomination of Ether.

When a contract is deployed, it is added as a transaction to the ledger and is given an address which can be used to call its functions once mined\cite{md-mp}. Since the blockchain is a segmented ledger, the contract cannot be modified and cannot be deleted. It is therefore crucial that a developer is certain of their code being correct\cite{yp}.

Flint is a type-safe language syntactically inspired by Swift, designed for enabling developers to write safe smart contracts targeting blockchain platforms. Many attacks on Ethereum smart contracts in recent history could have been prevented if they were written in a development environment with modern verification features and with a language that disallows common preventable programmer errors.

The language and reference compiler were originally created by Franklin Schrans as part of his final-year individual project at Imperial College in 2017-2018, supervised by Prof. Susan Eisenbach\cite{flint}. The compiler was designed for compiling example code snippets but was limited in the kinds of contracts that could be compiled due to bugs and incomplete features. Our goal for this project was to further extend the Flint language to improve safety, and increase its appeal and utility for authors of smart contracts.

The appeal to a smart contract developer has not changed between our project and Franklin's. Fundamentally we stayed true to the Flint philosophy, both in the strong preference for adapting design decisions from the Swift language for programmer familiarity and in overall operational safety. We developed our language extensions in a public GitHub repository, enabling anyone to inspect and audit our code. At the end of the project we will contribute our language changes back to the original project repository to enable all Flint developers and maintainers to benefit from our work, for free.
